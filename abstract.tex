\begin{abstract}
Este trabajo investiga los factores que influyen en el rendimiento académico de estudiantes de secundaria, utilizando datos del dataset Student Alcohol Consumption, centrándose en alumnos de matemáticas. A través de análisis estadísticos, se contrastan hipótesis sobre cómo variables como las relaciones románticas, el acceso a internet, las clases particulares y el consumo de alcohol afectan a las notas y la asistencia escolar.

Los resultados muestran que, aunque los alumnos con pareja tienen un mayor número medio de faltas, esta diferencia no es estadísticamente significativa. Las clases particulares incrementan la probabilidad de aprobar, mientras que el acceso a internet no afecta significativamente a la variabilidad de las notas. El consumo de alcohol los fines de semana no mostró un impacto significativo en las calificaciones. Además de estos factores, se comprobó que las notas de los alumnos suelen aumentar a lo largo del curso.

Finalmente, se desarrolló un modelo LOGIT para predecir la probabilidad de aprobar, identificando que las clases particulares y la asistencia a clase son los factores más relevantes, aunque el modelo presenta limitaciones en su precisión (69\% de accuracy).

\textbf{Palabras clave}: rendimiento académico, factores externos, análisis estadístico, modelo predictivo, educación secundaria.   
\end{abstract}
