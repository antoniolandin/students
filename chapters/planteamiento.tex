\chapter{Planteamiento del problema}

El problema central que abordamos es determinar si los factores estudiados tienen un efecto estadísticamente significativo en el rendimiento académico, medido a través de las notas finales (G3). Para ello, planteamos las siguientes preguntas de investigación:

\begin{enumerate}
    \item ¿El tener una relación romántica aumenta el número de faltas a clase?

    \item ¿Mejoran las notas de los estudiantes a lo largo del curso?

    \item ¿Las clases particulares incrementan la probabilidad de aprobar?

    \item ¿El acceso a internet afecta la variabilidad de las notas?

    \item ¿El consumo de alcohol los fines de semana repercute negativamente en las calificaciones?

    \item ¿Es posible predecir la probabilidad de aprobar basándose en estos factores?
\end{enumerate}

Para responder estas preguntas, aplicamos contrastes de hipótesis tanto paramétricos (Z-test, t-tests) como no paramétricos (Mann-Whitney, Levene), así como técnicas de modelado predictivo (regresión LOGIT).

El objetivo final de estas preguntas es comprender que factores que escapan al control del centro educativo influencian la nota de los alumnos para así poder diseñar estrategias que aumenten la nota de los estudiantes.
