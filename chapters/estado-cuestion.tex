\chapter{Estado de la cuestión}

El rendimiento escolar ha sido ampliamente estudiado en las últimas décadas, posiblemente por su gran impacto socio-económico en la vida de las personas. Diversos estudios, además de estudiar los factores que pueden controlan los centros educativos, se han centrado en los externos como el entorno familiar, el consumo de sustancias y la relaciones personales de los alumnos.

En este trabajo se seguirá esta línea de investigación, por lo que analizaremos conclusiones previas de otros estudios sobre los factores que recopila nuestro conjunto de datos.

\section{Factores externos que influyen en el rendimiento académico}

Numerosos estudios \cite{mendez2018clima}\cite{sanchez2015entorno} destacan la importancia del entorno familiar para las notas de los estudiantes. En nuestro conjunto de datos contamos con variables como el nivel educativo de los padres (\textit{Medu}, \textit{Fedu}), el estado de convivencia de los padres (\textit{Pstatus}), el apoyo familiar escolar (\textit{famsup}) e incluso la calidad de las relaciones familiares percibida por los alumnos (\textit{famrel}).

Una de las variables que se estudiaran en este trabajo será la de las clases particulares (variable \textit{paid}). En España el refuerzo escolar mediante clases particulares no ha hecho más que aumentar en las últimas décadas \cite{RunteGeidel2013}. Sin embargo, algunos estudios \cite{baker2001worldwide} \cite{wiseman2021does} identificaron una correlación negativa entre las clases particulares 1 a 1 y los logros de los alumnos de matemáticas, sugiriendo que se utiliza más como un remedio que como una solución real.

Uno de los fenómenos más disruptores a nivel social, especialmente en los jóvenes, es el auge de internet. Varios estudios \cite{ramos2017procrastinacion} \cite{vila2018rendimiento} afirman que la adicción y el uso problemático de internet están correlacionados con peor rendimiento académico. Aunque hay que tener en cuenta estudios que sugieren que el acceso a internet puede llegar a favorecer las medias de los alumnos \cite{garcia2022analisis}.

Otros de los factores que analizaremos será el consumo de alcohol de los alumnos. Estudios previos sugieren que el consumo de alcohol a edades tempranas podría afectar al rendimiento escolar así como aumentar el número de faltas en clase \cite{wagner2007alcohol} \cite{kovacs2008relacion}.

En nuestros datos tenemos constancia de si los alumnos son solteros o no. En relación con el consumo de alcohol, un estudio de la universidad de Georgia determinó que los alumnos que tenían más citas reportaban cuatro veces más abandono de los estudios y además se observó en ellos el doble de abuso de sustancias como el alcohol \cite{orpinas2013dating}.

En cuanto a la construcción de modelos para predecir la nota de los alumnos, existen varios estudios que utilizan técnicas de \textit{data mining} y \textit{machine learning} para predecir el rendimiento académico de los alumnos \cite{mengash2020using} \cite{cortez2008using}.

\section{Metodologías utilizadas en estudios previos}
