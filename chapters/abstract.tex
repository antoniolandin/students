\begin{abstract}
Este estudio analiza el impacto de factores socioacadémicos en el consumo de alcohol y el rendimiento estudiantil, utilizando un dataset de $n=395$ alumnos de educación secundaria. Mediante contrastes paramétricos (\textit{t-tests}, ANOVA) y no paramétricos (Mann-Whitney, Kruskal-Wallis), identificamos que:

\begin{itemize}
    \item Los estudiantes en relaciones románticas presentan mayor consumo los fines de semana (\textit{p} $< 0.05$, test U de Mann-Whitney).
    \item Existe una correlación negativa significativa entre el consumo de alcohol (\texttt{Walc}) y las notas finales (\texttt{G3}) ($\rho = -0.21$, \textit{p} $= 0.002$, Spearman).
    \item No hay diferencias significativas en ausencias (\texttt{absences}) por género (\textit{p} $= 0.18$, t-test para muestras independientes).
\end{itemize}

Los resultados destacan la necesidad de intervenciones educativas diferenciadas según el perfil estudiantil. El análisis incluye visualizaciones (QQ-plots, boxplots) y verificación de supuestos (normalidad, homocedasticidad) para garantizar robustez metodológica.
\end{abstract}
