\chapter{Introducción}

El rendimiento académico de los alumnos no se limita solo a lo que ocurre dentro del aula, sino que existen multiples factores externos que impactan en sus resultados escolares: su situación familiar, cuanto tiempo usan redes sociales, cuanto salen con los amigos o incluso si tienen pareja o no.\\
En este trabajo exploraremos que factores de la vida personal de los alumnos influyen en sus notas. Para ello trabajaremos con el dataset \textit{Student Alcohol Consumption}, el cual está disponible públicamente en \href{https://www.kaggle.com/datasets/uciml/student-alcohol-consumption}{Kaggle}. Los datos provienen de un estudio \cite{Cortez2008} el cual se realizó en dos escuelas secundarias portuguesas. En este, se recopilaron datos sobre las notas académicas de los alumnos en las asignaturas de matemáticas y portugués, además de información sobre la vida personal de cada estudiante. \\
En este análisis estadístico, nos centraremos en los datos correspondientes a los alumnos de matemáticas y analizaremos ciertos aspectos de su vida los cuales se asocian de forma significativa con sus niveles de asistencia en clase o sus calificaciones.
