% Reporte con fuente de tamaño 12
\documentclass[12pt]{report}

% Packages
\usepackage[spanish]{babel}
\usepackage{amsmath}
\usepackage{amssymb}
\usepackage{amsthm}
\usepackage{graphicx}
\usepackage{float}
\usepackage{listings}
\usepackage{hyperref}
\usepackage{url}
\usepackage{pgfgantt}
\usepackage[strings]{underscore}
\usepackage{natbib}
\usepackage{fontspec}

\setmainfont{Times New Roman}

\bibliographystyle{plain}

\hypersetup{
    colorlinks=true,
    linkcolor={black},
    filecolor=magenta,      
    urlcolor=cyan,
    pdftitle={Factores que afectan al rendimiento académico},
    pdfpagemode=FullScreen,
}

\setlength{\parindent}{0pt} % Remove paragraph indentation

\begin{document}
\begin{titlepage}
    \begin{center}
        \vspace*{1cm}
 
        \Large\textbf{Factores que afectan al rendimiento académico}
 
        \vspace{0.5cm}
            Trabajo final de inferencia estadística
        \vspace{1.5cm}
 
        \textbf{Antonio Cabrera Landín}
 
        \vfill
             
        Trabajo para el doble grado de\\
        Ingeniería del Software y Matemática Computacional\\
             
        \vspace{0.8cm}
      
        \includegraphics[width=0.4\textwidth]{figures/logo-u-tad.png}
             
        Inferencia Estadística\\
        U-tad\\
        España\\
        Mayo 2025
             
    \end{center}
 \end{titlepage}

\begin{abstract}
Este trabajo investiga los factores que influyen en el rendimiento académico de estudiantes de secundaria, utilizando datos del dataset Student Alcohol Consumption, centrándose en alumnos de matemáticas. A través de análisis estadísticos, se contrastan hipótesis sobre cómo variables como las relaciones románticas, el acceso a internet, las clases particulares y el consumo de alcohol afectan a las notas y la asistencia escolar.

Los resultados muestran que, aunque los alumnos con pareja tienen un mayor número medio de faltas, esta diferencia no es estadísticamente significativa. Las clases particulares incrementan la probabilidad de aprobar, mientras que el acceso a internet no afecta significativamente a la variabilidad de las notas. El consumo de alcohol los fines de semana no mostró un impacto significativo en las calificaciones. Además de estos factores, se comprobó que las notas de los alumnos suelen aumentar a lo largo del curso.

Finalmente, se desarrolló un modelo LOGIT para predecir la probabilidad de aprobar, identificando que las clases particulares y la asistencia a clase son los factores más relevantes, aunque el modelo presenta limitaciones en su precisión (69\% de accuracy).

\textbf{Palabras clave}: rendimiento académico, factores externos, análisis estadístico, modelo predictivo, educación secundaria.   
\end{abstract}


\tableofcontents

\listoffigures

\chapter{Introducción}

El rendimiento académico de los alumnos no se limita solo a lo que ocurre dentro del aula, sino que existen multiples factores externos que impactan en sus resultados escolares: su situación familiar, cuanto tiempo usan redes sociales, cuanto salen con los amigos o incluso si tienen pareja o no.\\
En este trabajo exploraremos que factores de la vida personal de los alumnos influyen en sus notas. Para ello trabajaremos con el dataset \textit{Student Alcohol Consumption}, el cual está disponible públicamente en \href{https://www.kaggle.com/datasets/uciml/student-alcohol-consumption}{Kaggle}. Los datos provienen de un estudio \cite{Cortez2008} el cual se realizó en dos escuelas secundarias portuguesas. En este, se recopilaron datos sobre las notas académicas de los alumnos en las asignaturas de matemáticas y portugués, además de información sobre la vida personal de cada estudiante. \\
En este análisis estadístico, nos centraremos en los datos correspondientes a los alumnos de matemáticas y analizaremos ciertos aspectos de su vida los cuales se asocian de forma significativa con sus niveles de asistencia en clase o sus calificaciones.


\chapter{Planteamiento del problema}

El problema central que abordamos es determinar si los factores estudiados tienen un efecto estadísticamente significativo en el rendimiento académico, medido a través de las notas finales (G3). Para ello, planteamos las siguientes preguntas de investigación:

\begin{enumerate}
    \item ¿El tener una relación romántica aumenta el número de faltas a clase?

    \item ¿Mejoran las notas de los estudiantes a lo largo del curso?

    \item ¿Las clases particulares incrementan la probabilidad de aprobar?

    \item ¿El acceso a internet afecta la variabilidad de las notas?

    \item ¿El consumo de alcohol los fines de semana repercute negativamente en las calificaciones?

    \item ¿Es posible predecir la probabilidad de aprobar basándose en estos factores?
\end{enumerate}

Para responder estas preguntas, aplicamos contrastes de hipótesis tanto paramétricos (Z-test, t-tests) como no paramétricos (Mann-Whitney, Levene), así como técnicas de modelado predictivo (regresión LOGIT).

El objetivo final de estas preguntas es comprender que factores que escapan al control del centro educativo influencian la nota de los alumnos para así poder diseñar estrategias que aumenten la nota de los estudiantes.


\chapter{Estado de la cuestión}

El rendimiento escolar ha sido ampliamente estudiado en las últimas décadas, posiblemente por su gran impacto socio-económico en la vida de las personas. Diversos estudios, además de estudiar los factores que pueden controlan los centros educativos, se han centrado en los externos como el entorno familiar, el consumo de sustancias y la relaciones personales de los alumnos.

En este trabajo se seguirá esta línea de investigación, por lo que analizaremos conclusiones previas de otros estudios sobre los factores que recopila nuestro conjunto de datos.

\section{Factores externos que influyen en el rendimiento académico}

Numerosos estudios \cite{mendez2018clima}\cite{sanchez2015entorno} destacan la importancia del entorno familiar para las notas de los estudiantes. En nuestro conjunto de datos contamos con variables como el nivel educativo de los padres (\textit{Medu}, \textit{Fedu}), el estado de convivencia de los padres (\textit{Pstatus}), el apoyo familiar escolar (\textit{famsup}) e incluso la calidad de las relaciones familiares percibida por los alumnos (\textit{famrel}).

Una de las variables que se estudiaran en este trabajo será la de las clases particulares (variable \textit{paid}). En España el refuerzo escolar mediante clases particulares no ha hecho más que aumentar en las últimas décadas \cite{RunteGeidel2013}. Sin embargo, algunos estudios \cite{baker2001worldwide} \cite{wiseman2021does} identificaron una correlación negativa entre las clases particulares 1 a 1 y los logros de los alumnos de matemáticas, sugiriendo que se utiliza más como un remedio que como una solución real.

Uno de los fenómenos más disruptores a nivel social, especialmente en los jóvenes, es el auge de internet. Varios estudios \cite{ramos2017procrastinacion} \cite{vila2018rendimiento} afirman que la adicción y el uso problemático de internet están correlacionados con peor rendimiento académico. Aunque hay que tener en cuenta estudios que sugieren que el acceso a internet puede llegar a favorecer las medias de los alumnos \cite{garcia2022analisis}.

Otros de los factores que analizaremos será el consumo de alcohol de los alumnos. Estudios previos sugieren que el consumo de alcohol a edades tempranas podría afectar al rendimiento escolar así como aumentar el número de faltas en clase \cite{wagner2007alcohol} \cite{kovacs2008relacion}.

En nuestros datos tenemos constancia de si los alumnos son solteros o no. En relación con el consumo de alcohol, un estudio de la universidad de Georgia determinó que los alumnos que tenían más citas reportaban cuatro veces más abandono de los estudios y además se observó en ellos el doble de abuso de sustancias como el alcohol \cite{orpinas2013dating}.

En cuanto a la construcción de modelos para predecir la nota de los alumnos, existen varios estudios que utilizan técnicas de \textit{data mining} y \textit{machine learning} para predecir el rendimiento académico de los alumnos \cite{mengash2020using} \cite{cortez2008using}.

\section{Metodologías utilizadas en estudios previos}


\chapter{Desarrollo, resolución, resultados e interpretación}


\chapter{Conclusiones}

Este estudio ha explorado diversos factores externos que influyen en el rendimiento académico de estudiantes de secundaria, utilizando técnicas estadísticas para contrastar hipótesis y un modelo predictivo para evaluar su impacto conjunto.

\section{Resultados del trabajo}

A continuación, se resumen las principales conclusiones derivadas de los análisis realizados:  

\begin{enumerate}
    \item \textbf{Relaciones románticas y asistencia a clase}

    Aunque los alumnos con pareja presentaron un mayor número medio de faltas (7,44 frente a 4,84 en solteros), el contraste de Mann-Whitney no mostró diferencias significativas al 5\% de significación. Esto sugiere que, a pesar de la percepción generalizada, tener pareja no afecta significativamente a la asistencia escolar. 
    \item \textbf{Evolución de las notas a lo largo del curso}

    El análisis confirmó que las notas mejoran significativamente del primer período (G1) al final (G3), respaldando la efectividad del sistema educativo evaluado.

    \item \textbf{Clases particulares y probabilidad de aprobar}

    Los alumnos que recibieron clases particulares tuvieron una tasa de aprobados mayor (71,82\%) frente a quienes no las tomaron (63,08\%). El contraste de proporciones mostró que esta diferencia es estadísticamente significativa al 5\%, lo que respalda la utilidad de las clases de refuerzo. Sin embargo, como ya hemos visto, estudios previos \cite{baker2001worldwide} \cite{wiseman2021does} dudan de la efectividad real de las clases particulares tradicionales.

    \item \textbf{Acceso a internet y variabilidad en las notas}

    El acceso a internet no mostró un impacto significativo en la dispersión de las calificaciones, según el contraste de Levene. Una hipótesis puede ser que en la escuela del conjunto de datos no está extendida la adicción a internet, ya que como hemos visto en otros estudios \cite{vila2018rendimiento}, el uso problemático de internet puede conllevar un impacto significativo en las notas de los alumnos.

    \item \textbf{Consumo de alcohol y rendimiento académico}

    A pesar de que el 62\% de los alumnos consumía alcohol los fines de semana, no se encontró una diferencia significativa en las notas medias respecto a los no consumidores. Este resultado contrasta con investigaciones previas que vinculan el consumo temprano de alcohol con bajo rendimiento \cite{wagner2007alcohol}.
\end{enumerate}

    Respecto al modelo LOGIT desarrollado existe mucho margen de mejora. Al realizar el contraste de significancia de sus parámetros, vimos que no elegimos las mejores variables para predecir si los alumnos aprueban o suspenden. Además, solo tuvimos en cuenta 4 de las 30 variables de las que disponemos en el dataset.   

\section{Reflexiones finales}

Los resultados subrayan que el rendimiento académico es multifactorial y no puede atribuirse únicamente a variables aisladas. Factores como las clases particulares y la asistencia a clase demostraron tener un impacto tangible, mientras que otros (relaciones románticas, internet o alcohol) no fueron concluyentes en este contexto. Esto nos alerta de no solo fijarnos en los factores internos de los centros educativos, sino además tener más en cuenta los externos para poder desarrollar soluciones académicas más efectivas

\section{Limitaciones y futuras líneas de investigación}

\begin{itemize}
    \item La muestra proviene de un contexto geográfico y cultural específico (Portugal), lo que limita la generalización.  
    \item Algunas de las variables cualitativas pueden estar sesgadas (por ejemplo cuando se les pregunta a los alumnos por sus relaciones familiares).
    \item Se podrían utilizar técnicas de \textit{machine learning} para predecir las notas de los alumnos.
\end{itemize}

La mayor limitación de este trabajo quizá sea la falta de extensión y de estudio de los datos. Solo hemos conseguido abarcar 5 de las 30 variables del conjunto de datos, además se podría indagar mucho más en análisis de cada una de ellas.

\section{Reflexión personal}

En lo personal, este trabajo ha sido un valioso acercamiento al mundo académico de la estadística y me ha servido para ver las aplicaciones reales de la inferencia estadística para acercarnos a las verdades de los problemas que afectan a nuestra era.


\bibliography{bibliografia}

\end{document}
